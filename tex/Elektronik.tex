% !TeX root = ../ausarbeitung.tex

\subsection{Chip}

Kernstück des Treibers ist eine Full-H-Bridge.Ich habe dafür nicht den im Paper TODOTODO empfohlenen LMD18200T von Texas Instruments, sondern den Infineon IFX9201SG verwendet, da dieser hinsichtlich der benötigten Spezifikationen sehr ähnlich zur Empfehlung ist, aber nur ca. ein Fünftel dessen kostet. Die Spezifikationen des Chips sind im Datenblatt zu finden. TODOTODO Die Spannungsversorgung des Chips geschieht über Pin 4. Pins 5 und 7 bilden den Ausgang des Chips. Die Pins 3, 8, 9 und 10 sind für die Serial-Verbindung des Chips vorgesehen. Da diese hier nicht benötigt wird, werden sie nicht angeschlossen. Pin 6 bildet die Erdung des Chips. Mit Pin 11 kann der Ausgang des Chips ausgeschaltet werden. Da dies nicht benötigt wird, wird auch dieser Pin geerdet. Die an Pin 2 anliegende Spannung gibt den Spannungsbereich des Hoch-Signals an und wurde mit einen an der Versorgungsspannung angeschlossenen 5V Spannungskonstanter verbunden. Für die Schaltlogik sind lediglich Pin 1 und 12 verwendet worden. Dabei gibt Pin 12 an, ob der Motor der Ausgang angetrieben werden soll. Ist darauf ein Hoch-Signal gelegt, wird auf den Ausgang Spannung gegeben, liegt ein Niedirg-Signal, wird keine Spannung gegeben. Pin 1 entscheidet, in welche Richtung die Spannung gegeben werden soll.


\subsection{Platine}

Dieser Chip erfordert den Verbau auf einer Platine. Zum Design dieser Platine habe ich die frei verfügbare Software "Kicad" verwendet. TODOTODO Zunächst entstand ein Prototyp, nach dem in Abb  TODOTODO ersichtlichen Schaltplan. Die Platine hat drei Anschlüsse. Mit J1 wird die Spannungsversorgung bezeichnet. Diese ist direkt mit Pin 4 des Chips verbunden, wobei noch der Kondensator C3 zu glätten verwendet wird. Für C3 wurde, entgegen der Beschreibung, statt eines 100nF- ein 1 \(\mu\)F-Kondensator verbaut. Die Spannungsversorgung wird außerdem mit dem Spannungskonstanter verbunden, wobei die Kondensatoren C1 (100 \(\mu\)F) und C2 (10 \(\mu\)F) zum Glätten vor bzw. hinter dem Konstanter verbaut sind. In dieser Version der Treibers wird mit dem Spannungskonstanter neben Pin 2 auch noch Pin 12 versorgt. Der Eingang J2 ist mit Pin 1 verbunden und legt die Antriebsrichtung fest. Die Ausgänge des Chips sind über den Ausgang der Platine J3 und das RC-Glied miteinander verbunden. Das RC-Glied besteht aus einem Widerstand R3 mit 200 \(\Omega\)und mehreren Kondensatoren C4-C8 in Parallel-Schaltung. Die Kondensatoren haben jeweils 47 \(\mu\)F und somit insgesammt 235 \(\mu\)F.Hiebei handelt es sich um Keramikkondensatoren, da sie bidirektional verwendet werden.
Da Pin 12 durch den Spannungskonstanter dauerhaft ein Hoch-Signal bekommt, liegt an den Ausgängen dauerhaft eine Spannung an. Das führt jedoch zu einem Problem. Da der Widerstand im RC-Glied dauerhaft unter Spannung steht, wird er sehr warm. Daraus resultierende Schwankungen des effektiven Ohm'schen Widerstands spiegelten sich in Schwankungen der Verzögerung wieder. Zudem dürfte auch der Widerstand der Spule im Motor der Festplatte durch die Erwärmung schwanken, was auch zu einer Unsicherheit der Verzögerung führt. Um das Problem zu umgehen habe ich ein zweites Design erstellt, bei der Ausgang durch einen zusätzlichen Signaleingang aktiviert wird.
Dieses ist in TODOTODO zu sehen.
Hier ist ein weiterer Anschluss hinzugekommen der jetzt den Naemen J3 bekommt und Pin 12 aktiviert. Zusätzlich wurden beiden Signal-Eingänge mit den nötigen Kontaktplatten versehen, um über einen Widerstand geerdet zu werden. Sollte es nötig sein, das Signal zu Terminieren, können diese Widerstände angebracht werden. Zur Zeit sind sie TODOTODO