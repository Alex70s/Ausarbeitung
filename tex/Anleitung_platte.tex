\subsection{Vorbereitung der Platte}
Die Vorbereitung der Platten ist einfach. Zunächst wird das Gehäuse entfernt. Da es bei i-Pod-Festplatten mit Spezialschrauben befestigt ist, welche nachher nicht mehr gebraucht werden, ist es am einfachsten diese aufzubohren. Unter dem Gehäuse befindet sich zunächst die PLatine, welche nicht benötigt wird und durch einfaches abziehen entfernt werden kann. Die eingentliche Platte kann durch aufbohren der Schraube in der Mitte der Halterung entfernt werden. Danach kann man beginnen die Festplatte zurecht zu sägen. Die Abmessung befinden sich im Anhang %TODO%
Es empfiehlt sich beim Sägen und dem anschließenden Feilen der Kanten den Magneten abzumachen um eine Verschmutzung durch (magnetische) Sägespähne zu vermeiden. Dieser ist aufgesteckt und hällt sich durch magnetische Kräfte am Gehäuse fest. 
Anschließend die Befestigung der Flagge. Diese sollte nach den Maßen in Abb %TODO%
aus dünnen Blech hergestellt und in den Arm der Festplatte eingesteckt werden. Dort kann sie mit Klebstoff, hier wurd EpoxydHartz verwendet, befestigt werden.
Um den Festplattenarm am Ende des Laufweges abzubremsen wurden kleine Quader aus "Sorbothane" %TODO%
Kurz vor dem Anschlagspunkt in der Laufweg geklebt. Es ist dabei darauf zu achten, dass keine anderen Komponenten der Festplatte mit geklebt werden und keine Klabstoffrükstände im Laufweg der Amrs zurückbleiben, an denen der Arm schleifen könnte.
Jetzt muss nur noch die eelektrische Komponente gefertigt werden. Es gilt die beiden Kontakte zu finden, an denen der Motor angeschlossen ist. Dazu sollten die Anschlüsse auf der Rückseite der Fesplatte lokalisiert und paarweise mit dem Multimeter hinsichtlich ihers Widerstands vermessen werden. An Paare, die einen Widerstand von ca. 10 Ohm aufweisen, sollten eine Gleichspannung von wenigen Volt angelegt werden, um zu testen,ob sie dem Motor antreiben. Üblicherweiße ist es eines der äußeren Paare. Sind die Kontakte identifiziert, können daran Kabel gelötet werden. 