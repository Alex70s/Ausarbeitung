\subsection{Wärmeresistenz}
Um zu Testen, wie gut die mit Epoxyd-Hartz befestigte Flagge durch die Absorption des Laser-Licht auftretende Erwärmung aushält, habe ich die Flagge mit einer Kerze angerust, um die Absorbtivität zu erhöhen und anschließend der Shutter in den blauen Laser eingebracht. Dort wurde er zehn Minuten mit ca. 130 mW beleuchtet. Danach habe ich mit dem Finger geprüft, ob die eine Hitzeentwicklung spürbar ist. Der Arm und die Flagge waren lediglich lauwarm, 

\subsection{Stopper}

Zunächst habe ich eine Festplatte ohne Stopper präpariert und mit dem ersten Treiber-Prototypen vermessen. Bei ich lies die Festplatten bei 12V und 1Hz laufen und betrachte das Signal. Man konnte deutlich sehen, dass es zu starken Überschwingern kam. Um diese zu verhindern, probierte ich verschiedene Stopper aus. Zunächst versuchte ich es mit Kunststoff-Schrauben. Ich bohrte zwei 2mm-Löcher in die Rückwand der Platte, sodass hindurhc gesteckte M2-Schrauben den Arm jeweils kurz vor dem Ende des Laufwegs stoppten. Dies konnte Überschwinger nur verhindern, wenn die SChrauben nicht alzu fest saßen. Daruas konnte man schließen, das die Endpositionen des Arms schwanken könnten, was zu einer hohen Unsicherheit der Verzögerungen führen dürfte. Daher fertigte ich noch drei weiteren Prototypen mit anderen Stoppern an. Davon wurden zwei Platten mit jeweils einem Stopper aus Strobopthan %TODO%
an jeweils unterschiedlichen Enden des Laufwegs und eine Platte mit Strobopthan an beiden Enden des Wegs ausgestattet. Es zeigte sich, dass Überschwinger am besten durch zwei Stopper zu verhindern waren.

\subsection{Änderung des Aufbaus}

Um zusehen, wie schnell die Shutter einen Strahl unterbrechen können, wurde eine Abbildungsoptik eingebaut. Zunächst wurde ein Größer Faserausgang verwendet, um nachfolgende Linsen besser auszuleuchten, dann wurden ein asphäriche Linse mit 1mm Brennweite auf einem Z-Verschieber eingebracht. Die Shutterflage sollte im Fokus dieser Linse liegen. Hinter den Shutter wurde eine weitere Linse mit 18mm-Brennweite eingebracht, um den Laser auf die Photodiode abzubilden. Dabei ließ sich die Schließzeit durch genaues Einstellen der Linse und somit des Fokus auf 5%TODO%
reduzieren.

\subsection{Hitzproblem}

Durch die starke Verkürzung der Schließzeit fiel die Schwankung des Schließverzögerung sehr stark auf. Um das genauer zu Untersuchen habe ich den Shutter mit einer Frequenz von 2Hz bei 15V für etwa 10min und dann nochmal 10min auf 7V laufen lassen. Dabie fiel auf, dass die Schließzeit eine Zeit lang stieg und dann in einem Bereich um einen Wert hängen blieb. Wurde danach die Spannung gesenkt, erhöhte sich die Schließzeit auf einen deutlich höheren Wert und fiel dann langsam ab, bis sie in einem Bereich um einen niedrigeren Wert hängen blieb. Eine Wiederholung mit einer Frequenz von 0,1Hz führte zu einem deutlich größeren Schwankungsbereich. %TODO%
Daraus und aus der Beobachtung, dass der im RC-Glied verbaute Widerstand sehr heiß wird, ließ die Folgerung zu, dass die Erwärmung des Widerstand zu einer Schwankung des ohm'chen Widerstand und daher zu einer Änderung der Schließverzögerung führt. Da sowohl Erwärmung und als auch Abkühlung mit einwirkt, sind bei höheren Frequenzen kleinere Schwankungen messbar. Um dieses Problem zu umgehen, habe ich ein zweites Treiberdesign entworfen, welches durch mögliche Abschaltung der Ausgänge zu weniger Wärmeentwicklung führen soll.

\subsection{Kurze Pulse}
Für die Verwendung des Shutters zur Erzeugung kurzer Puls ziegte sich ein Problem. Bei Pulslängen unter 200ms stieg die Schwankungen der Pulslänge stark an. Das liegt daran, dass der Shutter bei einer Bewegung gegen der Stopper schlägt und zurück springt. Der Arm stösst einige Mal an, bis er endgültig in Ruhe bleibt. Wird die Bewegungsrichtung geändert bevor der Arm steht, ist die Schließzeit davon abhängig, in welcher Position und welcher Bewegung der Arm zu dem Zeitpunkt ist. Daraus resultiert diese Unsicherheit. Um das zu umgehen, wurde zunächst ein anderes Shutterdesign probiert. Dabei wurde der Bewegungsbereich durch Positionierung der Stopper auf einen sehr kleinen Bereich eingeschränkt. Um entsprechend sichere Ergebnisse zu bekommen, habe ich drei Shutter so präpariert und mit verschieden langen Flaggen ausgestattet. Das Resultat der Tests bei unterschiedlichen Spannungen und Positionierungen des Lasers auf der Flagge ist das Auftreten nicht umgehbarer Überschwinger. Darauf hin wurde dieses Design wieder verworfen und eine "2-Shutter-Lösung" getestet. Dazu wurden zwei Shutter hintereinander in den Strahlgang gebracht. Dabei wurden die Shutter mit A und B bezeichent. Beide Shutter sind dabei so angebracht, dass die Halterungsstange nach oben gerichtet ist. Shutter B ist so positioniert, dass die Flagge bei liegendem Arm im Strahlgang ist. Der Strahl ist dabei auf die obere Kante der Flagge gerichtet. Shutter A ist so angebracht, dass die Flagge bei stehendem Arm im Strahlgang liegt. Dabei ist der Strahl auf die untere Kante der Flagge gebracht. Die Positionierung des Strahls auf der Flagge hat den einfachen Hintergrund, dass zum Einem der Arm zurück schwingt und daher eine Position an der dem Anschlagspunkt abgewandten Seite eine Öffnung des Strahl verhindert oder minimiert und zum Anderen dadurch der Arm die komplette Länge der Flagge durchläuft, bis der Strahl freigegeben wird und somit Geschwindigkeit aufbauen kann, wodurch die Schließzeit gesenkt wird. Zudem Sollten beide Flaggen im Strahl sehr nah aneinander sein, da so durch Fokusierung des Strahls zwichen den Shuttern eine Verkürzung der Öffnungs- und Schließzeiten entsteht. Die genaue Positionierung muss immer mit einer Beobachtung des Pulses einhergehen, da Überschwinger durchfeinjustierung verhindert wird. Genaueres dazu ist in %TODO%  zu finden.
Der Betrieb der Shutter wurde mittels PC und Pulsblaster gesteuert. Das verwendete Skript ist in %TODO% zu sehen.
Zunächst wurde der Enable getriggert, um die Shutter in Betrieb zu nehmen. Dann wurde Shutter A geschlossen. Da es dabei Überschwinger gibt, wird eine Zeit lang abgewartet, bis der Shutter den Strahl sicher schließt. Dabei geht es nicht darum, dass der Arm steht, da danach noch genug Zeit dafür gelassen wird. Es sollte lediglich der Strahl sicher geschlossen sein. Danach wurde Shutter B geöffnet. Hier muss länger gewartet werden, um sicherzustellen, dass der Arm in Ruhe ist. Nun können Pulse erzeugt werden. Durch Öffnung von Shutter A wird der Strahl freigegebne und durch Schließung von Shutter B wieder unterbrochen. Danach kann nach einer kurzen Nachlaufzeit da Enable-Signal abgeschaltet werden. 